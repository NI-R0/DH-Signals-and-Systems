\section{Einleitung}
Das System, welches im Folgenden behandelt wird hat die Übertragungsfunktion
\begin{equation*}
  G(s)=\frac{K}{T*s+1}
\end{equation*}\\
und mit den Werten $K=-25$ und $T=10$ ergibt sich daraus die neue Übertragungsfunktion
\begin{equation*}
  G(s)=\frac{-25}{10s+1} .
\end{equation*}\\
Um das System analysieren zu können, haben wir das System mit den entsprechenden Werten in Simulink simuliert und die Sprungantwort betrachet.\\
Erst danach haben wir mit Matlab gearbeitet und dort die entsprechenden Graphen plotten lassen. Mit Hilfe des Befehls \texttt{sys = tf([-25],[10 1])} haben wir unser System in Matlab erzeugt. Falls notwendig wird ein Sinus als Eingangssignal angenommen. %TODO: Stimmt das?
% Das System ist wie folgt zu klassifizieren:
%TODO: Klassifizierung