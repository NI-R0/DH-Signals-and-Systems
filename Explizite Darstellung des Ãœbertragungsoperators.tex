\section{Explizite Darstellung des Übertragungsoperators} %TODO: Überarbeiten
Die allgemeine explizite Form des Übertragungsoperators für ein lineares zeitinvariantes System ist (x(t) Formel Aufschrieb).
Für unser konkretes System kann man das mit der symbolik-Toolbox in Matlab berechnen: syms t und expm liefert ....
Die erzwungene Bewegung kann man aber nur dann explizit angeben, wenn man für u(tau) eine explizite Vorgabe hat. Das ist der Fall, wenn u(tau) beispielsweise ein Dirac-Impuls oder ... 
oder ein Sinusimpuls ist. In einem solchen Fall kann man das Integral analytisch ausrechnen. In meiner Arbeit wählen wir die Step-Funktion. 
Das heißt u(tau)=1 für tau>=0. Im konkreten erhalten wir die Formel, die ausgeplotet identisch ist, mit Darstellung in ...