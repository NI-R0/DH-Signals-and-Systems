\subsection{Physikalische Beschreibung/Umcodierung}
Unser System weist sowohl P- als auch T-Verhalten auf und ist ein Verzögerungsglied 1. Ordnung. In der Praxis lässt sich mit diesem System beispielsweise der Kühlvorgang in einer Tiefkühltruhe beschreiben.

\vspace*{0.5cm}
\begin{center}
  \begin{tabular}{l| |l|l|l}
    & Physikalisch & Normiert   & Systemtheoretisch \\\hline\hline
    & & & \\
    Eingang   & (Kühl-)Energie  & $Wh$ & $u$ \\
    Ausgang   & Temperatur      & °C                & $y$ \\
    Zustand   & Temperatur      & °C                & $x_{1}$ \\
    Parameter & \makecell[l]{Niedrigste Temperatur des\\ Kühlgerätes,\\Leistung des Kühlgerätes}                & °C, $\frac{BTU}{h}$    & $K$, $T_{1}$ \\
  \end{tabular}
\end{center}
\vspace*{0.5cm}
\noindent{}Die Wahl der Zeitkonstante $h$ ist naheliegend, da Kühlgeräte die Temperatur normalerweise nur langfristig (im Stundenbereich) ändern können.\\
Die Normierung und Festlegung der Einheiten ist wichtig, damit die Plots mit den richtigen Zeitskalen beschriftet werden können.

%Eine Normierung des Eingangs auf den Bereich 0 bis 100% ist häufig sinnvoll, da dann die statische Verstärkung angibt, auf welchem Endwert ich bei 100% (also max. Eingang) lande (bei P-Gliedern).

%ACHSEN DER PLOTS RICHTIG BESCHRIFTEN, damit die Einheiten richtig abgelesen werden können!!!! Beschriften: [t]=s oder t in s oder t/s oder "Zeit in Sekunden"

%Matlab Befehle: labelx/yaxis: entweder t/s oder t in s/sec./Sekunden oder Zeit in Sekunden

  %CHRIS:
  % \begin{tabular}{l|l|l|l}
        %     \centering
        %         0 & physikalische Darstellung & normierte Darstellung & systemtheoretische Darstellung  \\ \hline
        %         Eingang & Durchfluss Wasserhahn & 0-100\% & u \hline
        %         Ausgang & Höhe des Wassers & V/V_ges & y \hline
        %         Zustand & Höhe des Wassers & cfaef & x_1, x_2 ... je nach Ordnung \hline
        %         Parameter & Masse, Länge,Trägheitsmoment, Kapazität, Induktivität & T, $T_D$, D
        % \end{tabular}