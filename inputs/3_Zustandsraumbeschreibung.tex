\subsection{Allgemein}
Um die Zustandstaumdarstellung zu bestimmen, werden die Matrizen $A$, $B$, $C$ und $D$ benötigt. Diese stehen wie folgt mit dem System im Zusammenhang:
\begin{equation*}
  \dot\vec{x} = A\vec{x} + B\vec{u}
\end{equation*}
\begin{equation*}
  y = C\vec{x} + D\vec{u}
\end{equation*}
Die Zustandsübergangsfunktion beinhaltet die Systemmatrix $A$ und die Eingangsmatrix $B$ und gliedert sich in den Eigenvorgang sowie den erzwungenen Vorgang. Im Folgenden ist diese Übergangsfunktion dargestellt:
\begin{equation*}
  \varphi (t,t_0,x_0,u) = e^{A(t-t_0)}x(t_0)+\int_{t_0}^{t}e^{A\tau }Bu(t-\tau )d\tau
\end{equation*}
Der Übertragungsoperator bezieht zusätzlich die Ausgangsmatrix $C$ und die Durchgriffsmatrix $D$ mit ein.
\begin{equation*}
  \psi (t,t_0,x_0,u_{[t_0,t)}) = Ce^{A(t-t_0)}x(t_0)+\int_{t_0}^{t}Ce^{A\tau }Bu(t-\tau )d\tau + Du(t)
\end{equation*}
\vspace*{0.5cm}
Die Matrizen $A$, $B$, $C$, $D$ lassen sich in Matlab wie folgt generieren:\\
\hspace*{0.5cm}\texttt{sys = tf([-25],[10 1])}\\
\hspace*{0.5cm}\texttt{ss(sys)}



\subsection{Unser System}
Werden die obigen Befehle auf unser System angewendet, erhält man die Werte:
\begin{equation*}
  A = [-0.1] ; B = [2] ; C = [-1.25] ; D = 0
\end{equation*}
Aus diesen lässt sich die folgende Zustandsraumdarstellung herleiten:
\begin{equation*}
  \dot \vec{x} = [-0.1][x_1] + [2]u
\end{equation*}
\begin{equation*}
  y = [-1.25][x_1] + [0]u
\end{equation*}