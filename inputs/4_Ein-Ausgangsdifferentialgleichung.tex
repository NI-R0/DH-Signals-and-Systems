\subsection{Allgemein}
% Durch Überkreuzmultiplikation kann man von der Übertragungsfunktion auf die Eingangs- / Ausgangsdifferentialgleichung schließen. Von der Form:\\
Die Eingangs-/Ausgangsdifferentialgleichung kann aus der Übertragungsfunktion durch Rücktransformation aus dem Laplace-Bereich bestimmt werden:
\begin{equation*}
  G(s) = \frac{Y(s)}{U(s)} = \frac{K}{T_1*s+1}
\end{equation*}
Durch Überkreuzmultiplikation gelangt man zu:
\begin{equation*}
  Y(s)*(T*s+1) = U(s)*K
\end{equation*}
Durch Ausklammern erhält man:
\begin{equation*}
  T*Y(s)*s + Y(s) = U(s)*K
\end{equation*}
Die lineare Differentialgleichung wird durch Umwandlung mit Hilfe der inversen Laplace-Transformation ermittelt:
\begin{equation*}
  T* \dot y(t) + y(t) = K * u(t)
\end{equation*}
Zur Vereinheitlichung der Ableitungen $y(t)$ werden die Koeffizienten mit dem Buchstaben a dargestellt.
Auf der rechten Seite werden die Koeffizienten der Ableitungen von $u(t)$ mit $b_0$ und fortlaufend nummeriert:
\begin{equation*}
  a_1\dot y(t)+a_0y(t) = b_0u(t)
\end{equation*}
Die höchste Ableitung von $y$ wird vom Koeffizienten freigestellt, in dem alle Terme der Gleichung auch durch $a_1$ dividiert werden und nach der höchsten Ableitung aufgelöst wird:
\begin{equation*}
  \dot y(t)= \frac{1}{a_1}(-a_0y(t) +b_0u(t))
\end{equation*}

\subsection{Unser System}
Werden die obigen Schritte auf die Übertragungsfunktion unseres Systems angewendet, erhält man:
\begin{equation*}
  G(s)=\frac{Y(s)}{U(s)}=\frac{-25}{10s+1}
\end{equation*}
\begin{equation*}
  Y(s)*(10s+1) = -25*U(s)
\end{equation*}
\begin{equation*}
  10*Y(s)*s + Y(s) = -25*U(s)
\end{equation*}
\begin{equation*}
  10 \dot y(t) + y(t) = -25u(t)
\end{equation*}
Nach $\dot y(t)$ umformen:
\begin{equation*}
  \dot y(t)= \frac{1}{10}(-y(t)-25u(t))
\end{equation*}