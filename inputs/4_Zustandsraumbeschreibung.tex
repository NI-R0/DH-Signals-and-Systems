Mein System in neuer Zustandsraumdarstellung nach Transformation
-> Zusandsraumdarstellung
%Gewöhnliche Differenzialgleichungen eines Übertragungssystems

%Beschreibung linearer Systeme im komplexen Frequenzbereich

%Numerische Beschreibung linearer und nichtlinearer Systeme

Standard-Übertragungsfunktion eines 
Verzögerungsglied mit proportionalen Übertragungsverhalten.
\subsection*{Allgemein}

\begin{equation*}
  \ddot y(t)= \frac{1}{a_2}(-a_0\dot y(t)-a_1 y(t) +b_0u(t))
\end{equation*}

Eine Differenzialgleichung n-ter Ordnung benötigt zur Lösung n Integrationen. 
Nach dem Blockschaltbild zur Lösung der Differenzialgleichung 1. Ordnung ergibt sich eine Zustandsvariable als Ausgang der Integratore. 
Durch Substitution wird die Ableitung von $y(t)$ durch die Bezeichnung der Zustandsvariable $x(t)$ wie folgt eingesetzt: 
\begin{equation*}
  x_1(t) = y(t); x_2(t) = \dot y(t)
\end{equation*}

Damit lautet die Differentialgleichung mit der eingeführten neuen Bezeichnungder Zustandsvariable:

\begin{equation*}
  \ddot y(t)= \frac{1}{a_2}(-a_0 x_1(t)-a_1 x_2(t) +b_0u(t))
\end{equation*}

Daraus können die Zustandsgrößen gebildet werden.

\begin{equation*}
  \dot x_1 = \dot y = x_2
\end{equation*}

\begin{equation*}
  \dot x_2= \ddot y = -\frac{a_0}{a_2}x_1--\frac{a_1}{a_2}x_1+\frac{b_0}{a_2}u
\end{equation*}

Die Zustandsgrößen $x_1$ und $x_1$ bilden den sogenannten Zustandsvektor $\vec x$.

Diese Gleichungen werden als Vektordifferenzialgleichungen in Matizenform wie folgt geschrieben:
\[
  \begin{bmatrix}
    \dot x_1\\
    \dot x_2
  \end{bmatrix}
   =
  \begin{bmatrix}
     0                  & 1             \\
     -\frac{a_0}{a_2}   & -\frac{a_1}{a_2}
  \end{bmatrix}
  *
  \begin{bmatrix}
    x_1  \\
    x_2
  \end{bmatrix}
  +
  \begin{bmatrix}
    0               \\
    \frac{b_0}{a_2}
  \end{bmatrix}
  *u(1)
\]
 und die Ausgangsgleichung:
\begin{align*}
  \vec y(t) =
  \begin{bmatrix}
    1     & 0
  \end{bmatrix}
  *
  \begin{bmatrix}
    x_1     \\
    x_2
  \end{bmatrix}
\end{align*}

\subsection*{Unser System}

\begin{equation*}
  \dot y(t)= \frac{1}{10}(-y(t)-25u(t))
\end{equation*}

\begin{equation*}
  x_1 =y(t)
\end{equation*}

\begin{equation*}
  \dot y(t)= \frac{1}{10}(-x_1(t)-25u(t))
\end{equation*}

\begin{equation*}
  \dot x_1 = \dot y
\end{equation*}

\begin{equation*}
  \dot x_1= \frac{1}{10}(-x_1(t)-25u(t))
\end{equation*}

\begin{align*}
  \begin{bmatrix}
    \dot x_1
  \end{bmatrix}
  =
  \begin{bmatrix}
    -\frac{1}{10}
  \end{bmatrix}
  *
  \begin{bmatrix}
    x_1
  \end{bmatrix}
  +
  \begin{bmatrix}
    -\frac{25}{10}
  \end{bmatrix}
  *
  \begin{bmatrix}
    u(t)
  \end{bmatrix}
\end{align*}

\begin{align*}
    \vec y(t)
  =
  \begin{bmatrix}
    1
  \end{bmatrix}
  *
  \begin{bmatrix}
    x_1
  \end{bmatrix}
\end{align*}



%Strukturbild dargestllte Blockschaltbild
