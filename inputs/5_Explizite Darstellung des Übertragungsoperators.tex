%TODO: Überarbeiten
Die allgemeine explizite Form des Übertragungsoperators für ein lineares zeitinvariantes System sieht aus wie folgt:
\begin{equation*}
    \underline{x}(t) = e^{At}\underline{x_0}+\int_{0}^{t}e^{A(t- \tau )}bu(\tau )d\tau
\end{equation*}
\begin{equation*}
    y(t) = c^{T}x(t)+du(t)
\end{equation*}
Für unser konkretes System kann man das mit der Symbolik-Toolbox und folgenden Befehlen in Matlab berechnen:\\
\hspace*{0.5cm}\texttt{syms t}\\
\hspace*{0.5cm}\texttt{expm(A*t)}\\
Mit unserer Systemmatrix $A = [-0.1]$, dem Eingangsvektor $b = (-2.5)$, dem Ausgangsvektor $c = (1)$, dem Durchgangsfaktor $d = 0$ und dem Anfangswert $x_0 = 0$ erhält man:
\begin{equation*}
    \underline{x}(t) = \int_{0}^{t}-2.5*e^{\frac{\tau - t}{10}}*u(\tau )d\tau
\end{equation*}
\begin{equation*}
    y(t) = x(t)
\end{equation*}
In unserer Arbeit wählen wir die Step-Funktion für den Eingang, das heißt $u(\tau ) = 1$ für $\tau \geq 0$. Im konkreten Fall erhalten wir mit dem Befehl\\ \texttt{int(-2.5*expm((tau - t)/10), tau, 0, t)}\\ die Formel:
\begin{equation*}
    25*e^{\frac{-t}{10}} -25
\end{equation*}
Diese ist geplottet identisch mit der Darstellung in Abbildung 1.