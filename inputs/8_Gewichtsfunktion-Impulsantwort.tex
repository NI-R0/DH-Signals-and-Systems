%CHRIS: Antwort auf einen Delta-Impuls.
%Die Gewichtsfunktion kann aus der Übertragungsfunktion durch Ableiten erhalten werden. (Diagramm in Latex einbinden).
%Alternativ kann die Gewichtsfunktion aus der Übertragungsfunktion durch Differention gewonnen werden.
%Die Lösung von Matlab ist zwar die gleiche, aber in einer anderen Darstellung. Sollte Matlab einen zu langen Ausdruck ausgeben, kann der Befehl simplify(ans) verwendet werden, um den Ausdruck zu vereinfachen.
...
\subsection{Parametrische Darstellung} %NICHT SICHER
%Die Parametrische Darstellung der Gewichtsfunktion erhält man, indem man die Laplace-Rücktransformierte der Übertragungsfunktion bildet. Das macht man wie folgt: 
%syms s -> G(s)=bspw. 3*s/(1+9*s + 20*s^2) -> ilaplace(G(s)) -> t=[0;0.1;10] -> plot(t, Antwort der parametrischen Darstellung) (Da t ein Vektor ist, muss man beim Multiplizieren das Punktprodukt (Hadamad-Produkt) .* verwenden) (Alternativ kann man das mit dem fplot befehl machen, da spart man sich das Hadamad-Produkt) -> xlabel (Mit den Befehlen xlabel und ylabel kann man die Achsen beschriften) -> ylabel Gewichtsfunktion
%texttt{sys=tf([..][..])}
%texttt{syms s \\ ...}
...
\subsection{Nichtparametrische Darstellung}
\subsubsection{Plot mit Matlab}
% Durch den Befehl \texttt{impulse(sys)} öffnet MATLAB ein neues Fenster mit der passenden Sprungantwort h(s) zu G(s) und stellt diese folgendermaßen dar.

%Um die Gewichtsfunktion g(t) zu bestimmen, gebe ich den Befehl\\
%\texttt{ilaplace(G(s))}
%ein.
%Möchte ich mein Ergebnis g(t) weiterverwenden, dann definiere ich mir die symbolische Funktion\\
%\texttt{syms g(t)=ilaplace(G(s))}
...
\subsubsection{Plot mit Step-Funktion}
...
\subsubsection{Plot mit Simulink}
...