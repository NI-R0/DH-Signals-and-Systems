Um ein System zu beschreiben geht man im allgemeinen nach den folgenden Schritten vor:
\begin{enumerate}
    \item Detaillierungsgrad fürs Modell festlegen
    \item Physikalisches Modell erstellen
    \item Eingangs-, Ausgangs-, Zustandsgrößen festlegen (u,y,x,z)
    \item Physikalische Einheiten festlegen, ggf. Normierung in \% $\Longleftrightarrow$ Einheitenfreie Darstellung
    \item Analyse des Modells
    \begin{itemize}
        \item [a] Ruhelagen, evtl. linearisieren; Anfangswerte
        \item [b] Systemeigenschaften ermitteln
    \end{itemize}
    \item Entwurf
    \begin{itemize}
        \item [a] Ziele festlegen
        \item [b] Arbeitspunkte festlegen bzw. Trajektorien planen
        \item [c] Entwuft (Struktur, Bereich, Verfahren, Kriterium)
        \end{itemize}
\end{enumerate}