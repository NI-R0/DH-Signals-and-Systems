Natürlich kann die Umformung von der Zustandsraumbeschreibung in die Übertragungsfunktion auch mathematisch durchgeführt werden:
Gleichung 1:
\begin{equation}
	\dot{X}(t)=A*x(t)+B*u(t)
\end{equation}
wird transformiert in:
\begin{equation}
	s*X(s)=A*X(s)+B*U(s)
\end{equation}
Gleichung 2:
\begin{equation}
	y_{a}(t)=C*x(t)+D*u(t)
\end{equation}
wird transformiert in:
\begin{equation}
	Y_{a}(s)=C*x(s)+D*U(s)
\end{equation}
bei Gleichung 1 wird  das $x(s)$ isoliert:
\begin{equation}
	s*X(s)=A*X(s)+B*U(s)
\end{equation}
\begin{equation}
	X(s)*(sE-A)=B*U(s)
\end{equation}
wobei $E$ die Einheitsmatrix ist.
\begin{equation}
	X(s)=(sE-a)^{-1}U(s)
\end{equation}
Im nächsten Schritt wird das $x(s)$ der Gleichung 1 in das $x(s)$ der Gleichung 2 eingesetzt:
\begin{equation}
	Y_{a}(s)=C*(sE-a)^{-1}U(s)+D*U(s)
\end{equation}
Umgeformt ergibt es die Formel:
\begin{equation}
	G(s)=\frac{Y_{a}(s)}{U(s)}=C*(sE-A)^{-1}*B+D
\end{equation}
Falls man die Differentialgleichung erhalten möchten muss man die gewonnene Übertragungsfunktion in die Differentialgleichung umwandeln.
