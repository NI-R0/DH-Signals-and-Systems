Natürlich kann die Umformung von der Zustandsraumbeschreibung in die Übertragungsfunktion auch mathematisch durchgeführt werden:\\
Von den Ausgangsgleichungen:
\begin{equation*}
	\dot{x}(t) = Ax(t) + Bu(t)
\end{equation*}
\begin{equation*}
	y(t) = Cx(t) + Du(t) \text{, bzw. } y(t) = c^{T}\lambda \text{ (SISO), bzw. } y(t) = C\lambda \text{ (MIMO)}
\end{equation*}
gelangt man duch Laplace-Transormation zu:
\begin{equation*}
	sX(s) = AX(s) + BU(s)
\end{equation*}
\begin{equation*}
	\fbox{$Y(s) = c^{T}X(s) \text{ (SISO)}$} \text{ , bzw. } Y(s) = CX(s) \text{ (MIMO)}
\end{equation*}
Die Gleichungen von $Y(s)$ unterscheiden sich zwischen SISO- und MIMO-Systemen in dem Faktor vor $X(s)$. Da man $D=0$ wählen kann fällt der hintere Teil zudem weg. Im weiteren Verlauf wird der Einfachheit halber nur die oben durch einen Kasten markierte Formel eines SISO-Systems für die Rechnung verwendet. Die restlichen Schritte sind bei einem MIMO-System gleich.\\\\
In Gleichung 1 wird das $X(s)$ isoliert:
\begin{equation*}
	(sE-A)X(s) = BU(s)
\end{equation*}
\begin{equation*}
	X(s) = BU(s) * (sE-A)^{-1}
\end{equation*}
Eingesetzt in Gleichung 2 erhält man die Formel, um von der Zustandsraumdarstellung in eine DGL umzurechnen:
\begin{equation*}
	Y(s) = c^{T}(sE-A)^{-1}BU(s)
\end{equation*}
Um die inverse Matrix besser berechnen zu können kann die Formel umgeschrieben werden zu:
\begin{equation*}
	Y(s) = c^{T}\frac{adj(sE-A)}{det(sE-A)}BU(s)
\end{equation*}
Die Übertragungsfunktion lässt sich hieraus errechnen durch:
\begin{equation*}
	G(s) = \frac{Y(s)}{U(s)} = c^{T}\frac{adj(sE-A)}{det(sE-A)}B
\end{equation*}
Die Adjunkte in dieser Formel kann folgendermaßen errechnet werden:
\begin{equation*}
	adj(A) = cof(A)^{T}
\end{equation*}
Die einzelnen Elemente $A_{ij}$ der Kofaktormatrix ergeben sich wiederum aus
\begin{equation*}
	A_{ij} = (-1)^{i+j}*det_{ij}(A)
\end{equation*}
wobei $det_{ij}(A)$ durch Streichen der i-ten Zeile und j-ten Spalte entsteht.
Die Determinante einer 2x2-Matrix $A$ berrechnet man durch:
\begin{equation*}
	A = \begin{bmatrix}a&b\\c&d\end{bmatrix} -> det(A) = |A| = a*d - c*b
\end{equation*}
% Gleichung 1:
% \begin{equation*}
% 	\dot{x}(t)=A*x(t)+B*u(t)
% \end{equation*}
% wird transformiert in:
% \begin{equation*}
% 	s*X(s)=A*X(s)+B*U(s)
% \end{equation*}
% Gleichung 2:
% \begin{equation*}
% 	y_{a}(t)=C*x(t)+D*u(t)
% \end{equation*}
% wird transformiert in:
% \begin{equation*}
% 	Y_{a}(s)=C*x(s)+D*U(s)
% \end{equation*}
% bei Gleichung 1 wird  das $x(s)$ isoliert:
% \begin{equation*}
% 	s*X(s)=A*X(s)+B*U(s)
% \end{equation*}
% \begin{equation*}
% 	X(s)*(sE-A)=B*U(s)
% \end{equation*}
% wobei $E$ die Einheitsmatrix ist.
% \begin{equation*}
% 	X(s)=(sE-a)^{-1}U(s)
% \end{equation*}
% Im nächsten Schritt wird das $x(s)$ der Gleichung 1 in das $x(s)$ der Gleichung 2 eingesetzt:
% \begin{equation*}
% 	Y_{a}(s)=C*(sE-a)^{-1}U(s)+D*U(s)
% \end{equation*}
% Umgeformt ergibt es die Formel:
% \begin{equation*}
% 	G(s)=\frac{Y_{a}(s)}{U(s)}=C*(sE-A)^{-1}*B+D
% \end{equation*}

Falls man die Differentialgleichung erhalten möchten muss man die gewonnene Übertragungsfunktion in die Differentialgleichung umwandeln.
