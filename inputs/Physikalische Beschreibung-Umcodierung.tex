%CHRIS:
  % Welches technisches System passt zu meiner Systemklasse?
  % Unser System hat D-Verhalten. D-Verhalten zeigen z.B. Tachometer.
  % Schwingung??
  % Wie entsteht ein DTs Glied? Schaltung von...?

  %ICH:
  % PT$_{1}$ mit negativer Verstärkung
  % Systembeschreibung mit physikalischer Darstellung
  % Systembeschreibung in systemtheoretischer Darstellung

  % Mein System hat P-Verhalten. P deutet darauf hin 
  % P zeigt Raumtemperatur.
\textbf{Darstellung}
%Beispiel Badewanne
  %Eingang: Durchfluss Wasserhahn & 1/3600 m3/s oder V/Vges in %
  %Ausgang: Höhe des Wassers
  %Zustand: Höhe des Wassers
  %Parameter: Fläche & & & Ti

  %TODO Tabelle
  %CHRIS:
  % \begin{tabular}{l|l|l|l}
        %     \centering
        %         0 & physikalische Darstellung & normierte Darstellung & systemtheoretische Darstellung  \\ \hline
        %         Eingang & Durchfluss Wasserhahn & 0-100\% & u \hline
        %         Ausgang & Höhe des Wassers & V/V_ges & y \hline
        %         Zustand & Höhe des Wassers & cfaef & x_1, x_2 ... je nach Ordnung \hline
        %         Parameter & Masse, Länge,Trägheitsmoment, Kapazität, Induktivität & T, $T_D$, D
        % \end{tabular}

  %ICH:
  \begin{tabular}{l| |l|l|l}
    & Physikalisch & Normiert & Systemtheoretisch \\\hline\hline
    Eingang & x & x & $u$ \\
    Ausgang & x & x & $y$ \\
    Zustand & x & x & $x_{1}$, $x_{2}$ \\
    Parameter & Masse, etc. & x & $T_{1}$ \\   %, Länge, Kapazität, Induktivität, Trägheitsmoment
  \end{tabular}


% \begin{table}[]
%   \begin{tabular}{l | l | l | l}
%   & physikalische Darstellung & normierte Darstellung & sytematische Darstellung \\\hline
%   Eingang   & Durchfluss Wasserhahn     &     & u                        \\
%   Ausgang   & Füll höhe Wasser          &     & y                        \\
%   Zustand   & Höhe des Wassers          &     & x1,x2                    \\
%   Parameter & Fläche        &                 & %, Masse, Länge, Kapazität, Induktivität           
%   \end{tabular}
%   \caption{}
%   \label{tab:my-table}
%   \end{table}

%Ich habe mich für die Einheit/Zeitkonstanten ... entschieden.
%Die Normierung und Festlegung der Einheiten ist wichtig, damit ich bei den Plots weiß, in welchen Zeitskalen ich arbeite. Wenn ich alles auf Si normiere ist die Zeitkonstante üblicherweise in Sekunden/Minuten/Stunden.Ich habe mich entschieden für die Zeitkonstanten in s oder min oder h.
%Eine Normierung des Eingangs auf den Bereich 0 bis 100% ist häufig sinnvoll, da dann die statische Verstärkung angibt, auf welchem Endwert ich bei 100% (also max. Eingang) lande (bei P-Gliedern).

%ACHSEN DER PLOTS RICHTIG BESCHRIFTEN, damit die Einheiten richtig abgelesen werden können!!!! Beschriften: [t]=s oder t in s oder t/s oder "Zeit in Sekunden"

%Matlab Befehle: labelx/yaxis: entweder t/s oder t in s/sec./Sekunden oder Zeit in Sekunden