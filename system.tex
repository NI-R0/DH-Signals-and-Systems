\documentclass[
  ngerman
  ,12pt
  ,pdftex
]{article}

\usepackage{graphicx}
\usepackage{amsmath}
\usepackage{amssymb}
\usepackage{float}
\usepackage{listings}
\usepackage[ngerman]{babel}
\usepackage[utf8]{inputenc}
\usepackage[T1]{fontenc}

\hyphenation{Differential-gleichung}
\hyphenation{Über-tragungs-ope-ra-tors}
\hyphenation{E/A-Differential-gleichung}

\begin{document}

\begin{titlepage}
  \begin{center}
      {\Huge \textbf{Signale und Systeme - Systemanalyse}}\\[1.5cm]
      {\Large Analyse des Systems}\\[1cm]
      {\Huge PT$_{1}$ mit negativer Verstärkung}\\[7cm]
      {\large Matrikelnummer: \textbf{8809469}}\\[0.5cm]
      {\large Matrikelnummer: \textbf{6130555}}\\[0.5cm]
      {\large Kurs: TINF21B3}\\[0.5cm]
      {\large Abgabedatum 16.12.2022}
      \vfill
  \end{center}
\end{titlepage}
\newpage
\tableofcontents
\newpage

% INPUTS

\section{Einleitung}  %stimmt
Das System, welches im Folgenden behandelt wird hat die Übertragungsfunktion
\begin{equation*}
  G(s)=\frac{K}{T*s+1}
\end{equation*}\\
und mit den Werten $K=-25$ und $T=10$ ergibt sich daraus die neue Übertragungsfunktion
\begin{equation*}
  G(s)=\frac{-25}{10s+1} .
\end{equation*}\\
Um das System analysieren zu können, haben wir das System mit den entsprechenden Werten in Simulink simuliert und die Sprungantwort betrachet.\\
Erst danach haben wir mit Matlab gearbeitet und dort die entsprechenden Graphen plotten lassen. Mit Hilfe des Befehls \texttt{sys = tf([-25],[10 1])} haben wir unser System in Matlab erzeugt. Falls notwendig wird ein Sinus als Eingangssignal angenommen. %TODO: Stimmt das?
% Das System ist wie folgt zu klassifizieren:
%TODO: Klassifizierung

\section*{Vorgehen in der Systemtheorie}   %eingefügt
\input{inputs/VorgehenSystemthoerie.tex}

\section{Physikalische Beschreibung/Umcodierung}   %stimmt
\section{Physikalische Beschreibung/Umcodierung}
%CHRIS:
  % Welches technisches System passt zu meiner Systemklasse?
  % Unser System hat D-Verhalten. D-Verhalten zeigen z.B. Tachometer.
  % Schwingung??
  % Wie entsteht ein DTs Glied? Schaltung von...?

  %ICH:
  % PT$_{1}$ mit negativer Verstärkung
  % Systembeschreibung mit physikalischer Darstellung
  % Systembeschreibung in systemtheoretischer Darstellung

  % Mein System hat P-Verhalten. P deutet darauf hin 
  % P zeigt Raumtemperatur.
\textbf{Darstellung}
%Beispiel Badewanne
  %Eingang: Durchfluss Wasserhahn & 1/3600 m3/s oder V/Vges in %
  %Ausgang: Höhe des Wassers
  %Zustand: Höhe des Wassers
  %Parameter: Fläche & & & Ti

  %TODO Tabelle
  %CHRIS:
  % \begin{tabular}{l|l|l|l}
        %     \centering
        %         0 & physikalische Darstellung & normierte Darstellung & systemtheoretische Darstellung  \\ \hline
        %         Eingang & Durchfluss Wasserhahn & 0-100\% & u \hline
        %         Ausgang & Höhe des Wassers & V/V_ges & y \hline
        %         Zustand & Höhe des Wassers & cfaef & x_1, x_2 ... je nach Ordnung \hline
        %         Parameter & Masse, Länge,Trägheitsmoment, Kapazität, Induktivität & T, $T_D$, D
        % \end{tabular}

  %ICH:
  \begin{tabular}{l| |l|l|l}
    & Physikalisch & Normiert & Systemtheoretisch \\\hline\hline
    Eingang & x & x & $u$ \\
    Ausgang & x & x & $y$ \\
    Zustand & x & x & $x_{1}$, $x_{2}$ \\
    Parameter & Masse, etc. & x & $T_{1}$ \\   %, Länge, Kapazität, Induktivität, Trägheitsmoment
  \end{tabular}


% \begin{table}[]
%   \begin{tabular}{l | l | l | l}
%   & physikalische Darstellung & normierte Darstellung & sytematische Darstellung \\\hline
%   Eingang   & Durchfluss Wasserhahn     &     & u                        \\
%   Ausgang   & Füll höhe Wasser          &     & y                        \\
%   Zustand   & Höhe des Wassers          &     & x1,x2                    \\
%   Parameter & Fläche        &                 & %, Masse, Länge, Kapazität, Induktivität           
%   \end{tabular}
%   \caption{}
%   \label{tab:my-table}
%   \end{table}

%Ich habe mich für die Einheit/Zeitkonstanten ... entschieden.
%Die Normierung und Festlegung der Einheiten ist wichtig, damit ich bei den Plots weiß, in welchen Zeitskalen ich arbeite. Wenn ich alles auf Si normiere ist die Zeitkonstante üblicherweise in Sekunden/Minuten/Stunden.Ich habe mich entschieden für die Zeitkonstanten in s oder min oder h.
%Eine Normierung des Eingangs auf den Bereich 0 bis 100% ist häufig sinnvoll, da dann die statische Verstärkung angibt, auf welchem Endwert ich bei 100% (also max. Eingang) lande (bei P-Gliedern).

%ACHSEN DER PLOTS RICHTIG BESCHRIFTEN, damit die Einheiten richtig abgelesen werden können!!!! Beschriften: [t]=s oder t in s oder t/s oder "Zeit in Sekunden"

%Matlab Befehle: labelx/yaxis: entweder t/s oder t in s/sec./Sekunden oder Zeit in Sekunden

\section{Explizite Darstellung des Übertragungsoperators}    %stimmt
\section{Explizite Darstellung des Übertragungsoperators} %TODO: Überarbeiten
Die allgemeine explizite Form des Übertragungsoperators für ein lineares zeitinvariantes System ist (x(t) Formel Aufschrieb).
Für unser konkretes System kann man das mit der symbolik-Toolbox in Matlab berechnen: syms t und expm liefert ....
Die erzwungene Bewegung kann man aber nur dann explizit angeben, wenn man für u(tau) eine explizite Vorgabe hat. Das ist der Fall, wenn u(tau) beispielsweise ein Dirac-Impuls oder ... 
oder ein Sinusimpuls ist. In einem solchen Fall kann man das Integral analytisch ausrechnen. In meiner Arbeit wählen wir die Step-Funktion. 
Das heißt u(tau)=1 für tau>=0. Im konkreten erhalten wir die Formel, die ausgeplotet identisch ist, mit Darstellung in ...

\section{Zustandsraumbeschreibung (implizite Systembeschreibung)}   %stimmt
\section{Zustandsraumbeschreibung (implizite Systembeschreibung)}
Mein System in neuer Zustandsraumdarstellung nach Transformation
-> Zusandsraumdarstellung
%Gewöhnliche Differenzialgleichungen eines Übertragungssystems

%Beschreibung linearer Systeme im komplexen Frequenzbereich

%Numerische Beschreibung linearer und nichtlinearer Systeme

Standard-Übertragungsfunktion eines 
Verzögerungsglied mit proportionalen Übertragungsverhalten.
\subsection*{Allgemein}

\begin{equation*}
  \ddot y(t)= \frac{1}{a_2}(-a_0\dot y(t)-a_1 y(t) +b_0u(t))
\end{equation*}

Eine Differenzialgleichung n-ter Ordnung benötigt zur Lösung n Integrationen. 
Nach dem Blockschaltbild zur Lösung der Differenzialgleichung 1. Ordnung ergibt sich eine Zustandsvariable als Ausgang der Integratore. 
Durch Substitution wird die Ableitung von $y(t)$ durch die Bezeichnung der Zustandsvariable $x(t)$ wie folgt eingesetzt: 
\begin{equation*}
  x_1(t) = y(t); x_2(t) = \dot y(t)
\end{equation*}

Damit lautet die Differentialgleichung mit der eingeführten neuen Bezeichnungder Zustandsvariable:

\begin{equation*}
  \ddot y(t)= \frac{1}{a_2}(-a_0 x_1(t)-a_1 x_2(t) +b_0u(t))
\end{equation*}

Daraus können die Zustandsgrößen gebildet werden.

\begin{equation*}
  \dot x_1 = \dot y = x_2
\end{equation*}

\begin{equation*}
  \dot x_2= \ddot y = -\frac{a_0}{a_2}x_1--\frac{a_1}{a_2}x_1+\frac{b_0}{a_2}u
\end{equation*}

Die Zustandsgrößen $x_1$ und $x_1$ bilden den sogenannten Zustandsvektor $\vec x$.

Diese Gleichungen werden als Vektordifferenzialgleichungen in Matizenform wie folgt geschrieben:
\begin{align*}
  \begin{bmatrix}
    \dot x_1\\
    \dot x_2
  \end{bmatrix}
  \left =
  \begin{bmatrix}
     0                  & 1             \\
     -\frac{a_0}{a_2}   & -\frac{a_1}{a_2}
  \end{bmatrix}
  \left *
  \begin{bmatrix}
    x_1  \\
    x_2
  \end{bmatrix}
  \left +
  \begin{bmatrix}
    0               \\
    \frac{b_0}{a_2}
  \end{bmatrix}
  \left *u(1)
\end{align*}
 und die Ausgangsgleichung:
\begin{align*}
  \vec y(t) =
  \begin{bmatrix}
    1     & 0
  \end{bmatrix}
  \left *
  \begin{bmatrix}
    x_1     \\
    x_2
  \end{bmatrix}
\end{align*}

\subsection*{Unser System}

\begin{equation*}
  \dot y(t)= \frac{1}{10}(-y(t)-25u(t))
\end{equation*}

\begin{equation*}
  x_1 =y(t)
\end{equation*}

\begin{equation*}
  \dot y(t)= \frac{1}{10}(-x_1(t)-25u(t))
\end{equation*}

\begin{equation*}
  \dot x_1 = \dot y
\end{equation*}

\begin{equation*}
  \dot x_1= \frac{1}{10}(-x_1(t)-25u(t))
\end{equation*}

\begin{align*}
  \begin{bmatrix}
    \dot x_1
  \end{bmatrix}
  \left =
  \begin{bmatrix}
    -\frac{1}{10}
  \end{bmatrix}
  \left *
  \begin{bmatrix}
    x_1
  \end{bmatrix}
  \left +
  \begin{bmatrix}
    -\frac{25}{10}
  \end{bmatrix}
  \left*
  \begin{bmatrix}
    u(t)
  \end{bmatrix}
\end{align*}

\begin{align*}
    \vec y(t)
  \left =
  \begin{bmatrix}
    1
  \end{bmatrix}
  \left *
  \begin{bmatrix}
    x_1
  \end{bmatrix}
\end{align*}



%Strukturbild dargestllte Blockschaltbild


\section{Ein-/Ausgangsdifferentialgleichung}   %stimmt
\section{Ein-/Ausgangsdifferentialgleichung}
\subsection*{Allgemein}
\begin{equation*}
  G(s) = \frac{Y(s)}{U(s)} = \frac{K_\text{PT1}}{T_1+1}
\end{equation*}
Die zugehörige linare Differentialgleichung wird durch Umwandlung mit Hilfe der inversen Laplace-Transformation ermittelt:
\begin{equation*}
  T_1\frac{\mathrm d}{\mathrm dt}y(t)+y(t)=K_\text{PT1}u(t)
\end{equation*}
mit $T_1$ als System-Zeitkonstante und dem Verstärkungsfaktor $K_\text{PT1}$.

Zur Vereinheitlichung der Ableitungen $y(t)$ werden die Koeffizienten mit dem Buchstaben a dargestellt.
Für die rechte Seite der Ableitungen von $u(t)$ mit b und fortlaufend nummeriert:

\begin{equation*}
  a_1\dot y(t)+a_0y(t) = b_0u(t)
\end{equation*}

Die höchste Ableitung wird vom Koeffizienten freigestellt, in dem alle Terme der Gleichung auch durch $a_1$ dividiert werden und nach $\dot y(t)$ aufgelöst wird:

\begin{equation*}
  \dot y(t)= \frac{1}{a_1}(-a_0y(t) +b_0u(t))
\end{equation*}

\section*{Unser System}
\begin{equation*}
  G(s)=\frac{-25}{10s+1}
\end{equation*}
Die zugehörige linare Differentialgleichung wird durch Umwandlung mit Hilfe der inversen Laplace-Transformation ermittelt:
\begin{equation*}
  10\dot y(t)+y(t)= -25u(t)
\end{equation*}
Nach $\dot y(t)$ umformen:
\begin{equation*}
  \dot y(t)= \frac{1}{10}(-y(t)-25u(t))
\end{equation*}

\subsection*{Zustandsraumbeschreibung in Übertragungsfunktion} %eingefügt
\subsection*{Zustandsraumbeschreibung in Übertragungsfunktion}
Natürlich kann die Umformung von der Zustandsraumbeschreibung in die Übertragungsfunktion auch mathematisch durchgeführt werden:
Gleichung 1:
\begin{equation}
	\dot{X}(t)=A*x(t)+B*u(t)
\end{equation}
wird transformiert in:
\begin{equation}
	s*X(s)=A*X(s)+B*U(s)
\end{equation}
Gleichung 2:
\begin{equation}
	y_{a}(t)=C*x(t)+D*u(t)
\end{equation}
wird transformiert in:
\begin{equation}
	Y_{a}(s)=C*x(s)+D*U(s)
\end{equation}
bei Gleichung 1 wird  das $x(s)$ isoliert:
\begin{equation}
	s*X(s)=A*X(s)+B*U(s)
\end{equation}
\begin{equation}
	X(s)*(sE-A)=B*U(s)
\end{equation}
wobei $E$ die Einheitsmatrix ist.
\begin{equation}
	X(s)=(sE-a)^{-1}U(s)
\end{equation}
Im nächsten Schritt wird das $x(s)$ der Gleichung 1 in das $x(s)$ der Gleichung 2 eingesetzt:
\begin{equation}
	Y_{a}(s)=C*(sE-a)^{-1}U(s)+D*U(s)
\end{equation}
Umgeformt ergibt es die Formel:
\begin{equation}
	G(s)=\frac{Y_{a}(s)}{U(s)}=C*(sE-A)^{-1}*B+D
\end{equation}
Falls man die Differentialgleichung erhalten möchten muss man die gewonnene Übertragungsfunktion in die Differentialgleichung umwandeln.


\section{Zusammenhänge der Funktionen $G(s)$, $g(t)$, $h(t)$ und $\frac{G(s)}{s}$}   %stimmt
Die Ubergangsfunktion stellt den die Veränderung des Signals innerhalb des Systems als Funktion dar. Diese
kann durch Integrieren aus der Gewichtsfunktion (später dargestellt) gebildet werden. Dafur nutzt man wie
in den folgenden Abbildungen dargestellt den Laplace-Bereich, da nicht alle Funktionen einfach integriert
werden können.
\begin{figure}[h]
    \begin{center}
        \includegraphics[width=13cm]{image/Screenshot 2022-12-13 161708.jpg}
    \end{center}
    \caption{syms s t, $G(s)=\frac{-25}{10s+1}$, h=ilaplace(H),g=diff(h), $G(s)$=laplace(g)}
\end{figure}
\\


\section{Übertragungsfunktion/Sprungantwort}
\input{inputs/Übertragungsfunktion-Sprungantwort.tex}

\section{Gewichtsfunktion/Impulsantwort} %NICHT SICHER
\input{inputs/Gewichtsfunktion-Impulsantwort.tex}

\section{Frequenzgang}
\input{inputs/frequenzgang.tex}

\section{Pol-Nullstellen-Plot}
\input{inputs/pol-nullstellen-plot.tex}

\section{Statische Kennlinie}%TODO: HBOX
\input{inputs/statische-kennlinie.tex}


\end{document}



%Vorteil Übertragungsfkt/Laplace-Transformation: Komplizierte Operationen in der Analysis wie Faltung äußern sich im Laplace-Bereich durch einfache Multiplikation. Sofern man in der Klasse der linearen Zeitinvarianten Systeme (LTI) ist, sind die Laplace-Transformierten rationale Funktionen (Polynom/Polynom) -> Damit gehen Polynomdivision, Patialbruchzerlegung usw.

%TODO: Linearisierung:
%Nachdem man die Ruhelagen berechnet hat kann man um jede einzelne Ruhelage das System linearisieren. In der Folge entsteht ein lineares System, welches für kleine Abweichungen um die Ruhelage Gültigkeit hat. Man nennt dieses auch die Variationsgleichung oder einfach auch linearisiertes System. Das linearisierte System ist deutlich einfacher zu behandeln, da sich z.B. über Eigenwerte Aussagen zur Stabilität der Ruhelage treffen lassen. Vorgehen:
%Im Falle eines SiSo-Systems ist f(x) die Jacobi-Matrix an der Ruhelage und f_u der Eingangsvektor.